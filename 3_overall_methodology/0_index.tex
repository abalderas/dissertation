
% this file is called up by thesis.tex
% content in this file will be fed into the main document

%: ----------------------- introduction file header -----------------------
\begin{savequote}[50mm]
Historical methodology, as I see it, is a product of common sense applied to circumstances. 
\qauthor{Samuel E. Morison}
\end{savequote}


\chapter{Metodología para la evaluación de competencias genéricas}
\label{cha:Overall methodology}

% the code below specifies where the figures are stored
\ifpdf
    \graphicspath{{3_overall_methodology/figures/PNG/}{3_overall_methodology/figures/PDF/}{3_overall_methodology/figures/}}
\else
    \graphicspath{{3_overall_methodology/figures/EPS/}{3_overall_methodology/figures/}}
\fi


%------------------------------------------------------------------------- 

\cite{turing1950computing}

\section{Como voy a evaluar}

Metolodogía mixta

Cómo voy a evaluar roles, momentos, actividad, ... lo que sea.

Explicar el DSL

DSL: herramienta de investigación en evaluaciones. Esta herramienta ayuda al investigador a formalizar la evaluación.

Enfocar la metodología a que no es una metodología para evaluar, sino para diseñar evaluaciones. Diseñador de evaluaciones.

Explicar qué pasos tiene que seguir un diseñador de evaluaciones para hacer sus evaluaciones.

Dentro de los resultados obtenemos dos cosas:
1. Diseño
2. Lo que el diseñador nos indicó que no pudo hacer

Para que esto sea posible falta la herramienta informática

\section{Evolución herramientas}

AMW --> EvalCourse --> EvalSim

\section{Metodologia de desarrollo}

DSL? Ing. Dirigida x modelo?



% ----------------------------------------------------------------------

