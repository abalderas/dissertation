%---------------------------------------------------------------
% Macros
% version 3 by Igor Ruiz-Agundez 2011
% version 2 by Jakob Suckale 2007
% version 1 by Harish Bhanderi 2002
%---------------------------------------------------------------

% This file contains macros that can be called up from connected TeX files
% It helps to summarise repeated code, e.g. figure insertion (see below).



%---------------------------------------------------------------
% Figures
%---------------------------------------------------------------


% Makes the \InsertFig macro compatible both with one or two columns
\makeatletter
\newlength \figwidth
\if@twocolumn
  \setlength \figwidth {\columnwidth}
\else
  \setlength \figwidth {\textwidth}
\fi
\makeatother

% \InsertFig allows inserting figures
% Parameters
% 1 --> Filename
% 2 --> Label for referencing
% 3 --> Title describing the figure (caption)
% 4 --> Description of the figure
% 5 --> Figure width, range [0,1]. If parameter is left blank the figure size is not change
% 6 --> Any other option for \includegraphics
% Usage:
% \InsertFig{}{}{}{}{}{}
%
\newcommand{\InsertFig}[6]{%
	\ifthenelse{\isempty{#5}}%
	{% if #1 is empty
		\begin{figure}[htbp!]
		\centering
		\includegraphics[#6]{#1}%
		\caption{#3}{\textbf{#4}}
		\label{#2}
		\end{figure}    
	}
	{% if #1 is not empty
		\begin{figure}[htbp!]
		\centering
		\includegraphics[width=#5\figwidth,#6]{#1}%
		\caption{#3}{\textbf{#4}}
		\label{#2}
		\end{figure}
	}
}

%% Simple version of \InsertFig
%\newcommand{\InsertFig}[5]{
%  \begin{figure}[htbp]
%   	\centering
%    \includegraphics[width=#4\textwidth,#5]{#1}%
%    \caption{#3}
%    \label{#2}
%  \end{figure}
%}



% insert a centered figure with caption
% parameters 1:filename, 2:label, 3:title, 
\newcommand{\figuremacro}[3]{
	\begin{figure}[htbp]
		\centering
		\includegraphics[width=1\textwidth]{#1}
		\caption[#3]{\textbf{#3}}
		\label{#2}
	\end{figure}
}


% insert a centered figure with caption and description
% parameters 1:filename, 2:label, 3:title, 4:description
\newcommand{\figuremacroD}[4]{
	\begin{figure}[htbp]
		\centering
		\includegraphics[width=1\textwidth]{#1}
		\caption[#3]{\textbf{#3} - #4}
		\label{#2}
	\end{figure}
}

% insert a centered figure with caption and description AND WIDTH
% parameters 1:filename, 2:label, 3:title, 4:description, 5: textwidth
% textwidth 1 means as text, 0.5 means half the width of the text
\newcommand{\figuremacroDW}[5]{
	\begin{figure}[htbp]
		\centering
		\includegraphics[width=#5\textwidth]{#1}
		\caption[#3]{\textbf{#3} - #4}
		\label{#2}
	\end{figure}
}

% inserts a figure with wrapped around text; only suitable for NARROW figs
% o is for outside on a double paged document; others: l, r, i(inside)
% text and figure will each be half of the document width
% note: long captions often crash with adjacent content; take care
% in general: above 2 macro produce more reliable layout
\newcommand{\figuremacroN}[3]{
	\begin{wrapfigure}{o}{0.5\textwidth}
		\centering
		\includegraphics[width=0.48\textwidth]{#1}
		\caption[#2]{{\small\textbf{#2} - #3}}
		\label{#1}
	\end{wrapfigure}
}




% Estas definiciones son para el comando \InsertFigBox
\newlength{\anchoFigura}
\newlength{\anchoFloat}
\addtolength{\fboxsep}{2\fboxsep}
%\renewcommand{\capfont}{\normalfont\normalcolor\sffamily\small}
%\renewcommand{\caplabelfont}{\normalfont\normalcolor\sffamily\bfseries\small}

% El comando \InsertFigBox nos permite insertar figuras en un marco
% Los parametros son:
% 1 --> Fichero de la imagen
% 2 --> Etiqueta (label) para referencias
% 3 --> Texto a pie de imagen
% 4 -> Porcentaje del ancho de página que ocupará la figura (de 0 a 1)
% 5 --> Opciones que queramos pasarle al \includegraphics
\newcommand{\InsertFigBox}[5]{%
  \setlength{\anchoFloat}{#4\textwidth}%
  \addtolength{\anchoFloat}{-4\fboxsep}%
  \setlength{\anchoFigura}{\anchoFloat}%
  \begin{figure}%
    \begin{center}%
      \Ovalbox{%
        \begin{minipage}{\anchoFloat}%
          \begin{center}%
            \includegraphics[width=\anchoFigura,#5]{#1}%
            \caption{#3}%
            \label{#2}%
          \end{center}%
        \end{minipage}
      }%
    \end{center}%
  \end{figure}%
}



%---------------------------------------------------------------
% Misc
%---------------------------------------------------------------

% predefined commands by Harish
\newcommand{\PdfPsText}[2]{
  \ifpdf
     #1
  \else
     #2
  \fi
}


%---------------------------------------------------------------
% Locales
%---------------------------------------------------------------


%%
%% Para quitar traducciones raras (Cuadros)
%% A de usarse cada vez que se seleccione el idioma
%%
\newcommand{\MejorarTraducciones}{%
       \renewcommand{\listtablename}{Índice de tablas}
       \renewcommand{\tablename}{Tabla}
       \renewcommand{\lstlistingname}{Lista}
}%



%---------------------------------------------------------------
% Source code
%---------------------------------------------------------------


%%
%% Para escribir extractos de codigo
%%
%% Las tabulaciones se substituyen por dos espacios
%\fvset{tabsize=2}
%% Creamos un nuevo environment de fancyvrb para los ejemplos enmarcados
%\DefineVerbatimEnvironment{VerbEj}{BVerbatim}{fontsize=\small,samepage=true,commandchars=\\\{\}}
%% Colo de fondo
%\definecolor{grisfondo}{gray}{0.9}
%% Environment para extractos de codigo
%\newenvironment{codigo}%
%{\VerbatimEnvironment\begin{Sbox}\begin{VerbEj}}%
%{\end{VerbEj}\end{Sbox}\setlength{\fboxsep}{8pt}\begin{center}\fcolorbox{black}{grisfondo}{\TheSbox}\end{center}}
%
%% Otro formato más bonito para código fuente
%\newcommand{\codigofuente}[3]{%
%  \lstinputlisting[language=#1,caption={#2}]{#3}%
%}



