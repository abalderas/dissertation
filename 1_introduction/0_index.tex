
% this file is called up by thesis.tex
% content in this file will be fed into the main document

%: ----------------------- introduction file header -----------------------


\begin{savequote}[50mm]
The beginning is the most important part of the work. 
\qauthor{Plato}
\end{savequote}

\chapter{Introducción}
\label{cha:Introduction}

% the code below specifies where the figures are stored
\ifpdf
    \graphicspath{{1_introduction/figures/PNG/}{1_introduction/figures/PDF/}{1_introduction/figures/}}
\else
    \graphicspath{{1_introduction/figures/EPS/}{1_introduction/figures/}}
\fi


%------------------------------------------------------------------------- 

La universidad es una institución impulsora de los cambios de todo tipo a los que la sociedad actual debe hacer frente. En el contexto del Espacio Europeo de Educación Superior~\footnote{http://www.eees.es/} e influenciado por la situación  actual de  la sociedad, sus instituciones sociales y políticas, la universidad se encuentra en el foco de las reformas para alcanzar la convergencia a nivel europeo. Bajo estas premisas, las competencias, las tareas y su evaluación se sitúan como los ejes del currículum universitario ~\cite{zabala2005espacio}.

Para que la empleabilidad de los graduados satisfaga las necesidades de Europa, el enfoque de las competencias juega un papel fundamental en este nuevo paradigma  ~\cite{communique2012making}. Los graduados deben adquirir y demostrar competencias genéricas, así como estar al día con el conocimiento específico de la materia a fin de ser capaces de satisfacer las necesidades de la sociedad y el mercado laboral.

En este contexto es fundamental el concepto de competencias como base para los resultados de aprendizaje. Las competencias representan una combinación dinámica de conocimientos, comprensión, habilidades y capacidades. El fomento de las competencias es el objetivo de los programas educativos, debiendo éstas ser desarrolladas en viarias unidades del curso y evaluadas en diferentes etapas ~\footnote{http://www.unideusto.org/tuningeu/competences.html}.

Las competencias se dividen en específicas y genéricas.

% Correo Juanma 15 enero: La parte original de la tesis de Antonio es el DSL, que mapea las queries de evaluación a queries sobre la BD del sistema de información que recoge evidencias, para "explorar" posibilidades de fórmulas/métodos con que expresar la evaluación de competencias, expresadas en ese DSL.





% ----------------------------------------------------------------------

