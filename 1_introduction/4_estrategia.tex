
\section{Estrategia de investigación}
\label{sec:Estrategia}

%Research strategic: Oates 2006
%- SLR
%- Cuestionario

%Ver tesis que me ha dejado Juanma

En esta sección primero se justifica y describe la estrategia de investigación que se empleó en esya tesis. En segundo lugar se explica cómo se puso en práctica. Para la explicación dividiremos esta sección en dos subsecciones:

\begin{itemize}
\item Diseño y creación
\item Esquema de la estrategia de investigación
\end{itemize}

\subsection{Diseño y creación}

Como se comentó al principio, el objetivo y contribución más importante de esta tesis es evaluar a los estudiantes en el desempeño de sus competencias genéricas mediante indicadores procedentes de los registros de actividades de aprendizaje. Para alzanzar este objetivo habrán de completarse dos fases:

\begin{itemize}
\item Definir un conjunto de indicadores (O1)
\item Validar estos indicadores (O2)
\end{itemize}

Para poder validar los indicadores y permitir a los docentes diseñar sus evaluaciones debemos construir las herramientas apropiadas. Por ello, esta tesis incluye el desarrollo de dichas aplicaciones.  En este sentido, la estrategia de investigación de diseño y creación seguida se puede definir de la siguiente manera:

\bigskip
\textbf{Combinación de una metodología de desarrollo de sistemas y una metodologia de investigación basada en una o más estregias de investigación que utilizan uno o varios métodos de generación de datos (Oates, 2006, Capítulo 8).}
\bigskip


En cuanto a la metodología de investigación de esta tesis: cuestionarios? Entrevistas? ... Ver Oates.

\bigskip
En cuanto a la metodología de desarrollo de sistemas ...

\subsection{Esquema de la estrategia de investigación}
