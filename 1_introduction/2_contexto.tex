
\section{Contexto}
\label{sec:contexto}



 % Correo Juanma 15 enero: La parte original de la tesis de Antonio es el DSL, que mapea las queries de evaluación a queries sobre la BD del sistema de información que recoge evidencias, para "explorar" posibilidades de fórmulas/métodos con que expresar la evaluación de competencias, expresadas en ese DSL.


% ----------------------------------------------------------------------

%Esta tesis ha sido escrita como parte de mi trabajo dentro del grupo de investigación Software Process Improvement and Formal Methods (SPI\&FM), perteneciente a la Universidad de Cádiz (UCA). Además, todos los experimentos fueron llevados a cabo en esta universidad. A continuación, se describirán brevemente tanto la universidad como el grupo de investigación:

%\subsection{Universidad de Cádiz (UCA)}

%La Universidad de Cádiz (UCA) es una universidad española ...

%\subsection{Software Process Improvement and Formal Methods (SPI\&FM)}

%El Grupo Software Process Improvement and Formal Methods (SPI\&FM) fue ...

La idea de que la educación pueda ser tratada como una \emph{ciencia del diseño} viene de la década de los 90, con la ambición de llevar la investigación educacional de los laboratorios a la práctica. Para hacer eso, los investigadores educacionales tenían que enfrentarse a \emph{la complejidad de las situaciones del mundo real y su resistencia al control experimental}~\cite{collins2004design}. La enseñanza se considera una ciencia, pues los investigadores educacionales investigan sobre ella, mientras que los profesores en general no investigan, sino que simplemente desarrollan y comparten teorías y explicaciones basadas en su propia experiencia~\cite{laurillard2012teaching}. Además, cuando un profesor o investigador educacional adopta un método innovador para implementarlo en sus clases, este es absorbido por el proceso normal de enseñanza, de forma que la implementación real puede convertirse en algo muy diferente del diseño original, es decir, hay un abismo entre las investigación y la práctica en la educación formal~\cite{anderson2012design}. 

La metodología de \emph{investigación del diseño} (DBR, del inglés, design-based research) fue concebida para solucionar esta separación entre la teoría y la práctica en la investigación. El DBR, cuyo método práctico principal es el experimento de diseño, es un enfoque de investigación mixto interdisciplinar que se lleva a cabo directamente en el área en la que se aplica y que enriquece también el conocimiento teórico de dicho área~\cite{reimann2011design}. DBR no es una metodología clásicamente experimental, sino iterativa, que se basa en ir refinando progresivamente el diseño inicial basado en la teoría. Según el análisis realizado por Terry Andersen y Julie Shattuck~\cite{anderson2012design}, las características que un estudio DBR de calidad en la educación debe tener son las siguientes:

\begin{itemize}
\item \emph{Contexto educativo real}: tener lugar en un contexto educativo real avala la validez de la investigación y asegura que los resultados puedan ser efectivamente utilizados para evaluar, informar y mejorar la práctica en, al menos, este contexto y probablemte en otros.
\item \emph{Enfocado en el diseño y prueba de una intervención significativa}: la selección y la creación de una intervención es una tarea colaborativa que atañe a investigadores y profesores. La creación comienza con un preciso análisis del contexto local; se basa en la literatura relevante, en la teoría y en las prácticas de otros contextos; y se diseña específicamente para solventar un problema o aportar una mejora en la practica. La intervención podría ser, por ejemplo, una actividad de aprendizaje, un tipo de evaluación, la introducción de una actividad administrativa (como un cambio en las vacaciones) o una intervención tecnológica. % the intervention may be a TYPE OF ASSESSMENT (el nuestro!)
\item \emph{Empleo de métodos mixtos}: las intervenciones DBR implican la aplicación conjunta de diferentes métodos mediante el empleo de una variedad de herramientas y técnicas de investigación. Los investigadores eligen, utilizan y combinan unos métodos u otros en función de sus necesidades.
\item \emph{Múltiples iteraciones}: la práctica del diseño suele implicar la creación y prueba de prototipos, refinamiento iterativo y la continua evolución del diseño, de la misma forma que ocurre en otros conocidos procesos de diseño como son, por ejemplo, la fabricación de coches o la moda.
\item \emph{Asociación colaborativa entre investigadores y profesores}: por una lado, los profesores suelen estar demasiado ocupados y no tienen experiencia para dirigir una investigación rigurosa. Por otro lado, los investigadores suelen carecer de conocimiento de la complejidad cultural, de la tecnología, de los objetivos y de las políticas de un sistema educativo que les permita crear y medir eficientemente el impacto de una intervención. Por tanto, se requiere una asocación para el estudio.
\item \emph{Evolución de los principios de diseño}: El diseño evoluciona desde y hacia la elaboración de principios de diseño, patrones y teorías funcionales. Estos principios no son diseñados para crear principios o teorías que tengan el mismo efecto en cualquier contexto, sino que sirven para ayudarnos en la comprensión del contexto y la intervención, y nos ayuden para ajustar ambos y así maximizar el aprendizaje.  El desarrollo de principios de diseño prácticos es una parte fundamental del DBR, y pone en desventaja a aquellos tipos de investigación que unilateralmente comienzan con las pruebas en clase y después desaparecen con el investigador una vez que el experimento ha concluido.
\item \emph{Comparación con la investigación-acción}: Tanto los profesores como los investigadores encuentran a menudo confuso diferenciar entre DBR e investigación-acción. Sin embargo, aunque ambas metodologías se sitúan dentro del campo de la investigación aplicada, difieren en características principales. Mientras que la investigación-acción se concibe principalmente para alcanzar una serie de objetivos a nivel local, en DBR se pretende también evolucionar a nivel teórico, maximizando la generalización y el entendimiento en la comprensión de aplicaciones prácticas. Además, la investigación-acción es llevada a cabo normalmente por un solo profesor, por lo que no se beneficia de la experiencia y la energía que caracterizan a los equipos de investigación y diseño DBR.
\item \emph{Repercusión en las prácticas}: El DBR no debe avanzar únicamente en el campo teórico, sino que para demostrar y justificar su valor real deberá ser además implementado en un contexto de estudio local.
\end{itemize}