
% this file is called up by thesis.tex
% content in this file will be fed into the main document

%------------------------------------------------------------------------- 

\section{Motivación}
\label{sec:Motivation}

\subsection*{Las competencias genéricas en el marco actual}

El papel de la universidad como institución impulsora de los cambios de todo tipo a los que la sociedad actual debe hacer frente es fundamental. En el contexto del Espacio Europeo de Educación Superior~\footnote{http://www.eees.es/} e influenciado por la situación actual de  la sociedad, sus instituciones sociales y políticas, la universidad se encuentra en el foco de las reformas para alcanzar la convergencia a nivel europeo. En este marco, son las competencias, las tareas y su evaluación los pilares en los que se basa el nuevo currículum universitario~\cite{zabala2005espacio}.

Para que la empleabilidad de los nuevos egresados satisfaga las necesidades del mercado laboral europeo, las competencias juegan un papel fundamental~\cite{communique2012making}. Las aptitudes y habilidades que la sociedad demandará a los futuros profesionales constituyen pilares básicos a tener en cuenta en el diseño de las estrategias educativas~\cite{de2010project}. Los nuevos egresados deben adquirir y demostrar competencias, además de dominar el conocimiento específico de la materia.

Por tanto, en lo que a la evaluación se refiere, podemos decir que el foco de interés se centra ahora en cómo evaluar a los estudiantes en el desempeño de sus competencias. Proyectos como el \emph{Tuning Educational Structures in Europe}, apoyado por el Lifelong Learning Program de la Unión Europea, muestran la importancia de utilizar el concepto de competencia como base para los resultados de aprendizaje. Las competencias de aprendizaje son habilidades que un alumno ha de ser capaz de demostrar una vez que termina su formación. Estas competencias de aprendizaje se dividen en dos grupos: específicas y genéricas. Competencias específicas son aquellas relacionadas directamente con la utilización de conceptos, teorías o habilidades propias de un área en concreto, mientras que las competencias genéricas son habilidades, capacidades y conocimientos que cualquier estudiante debería desarrollar independientemente de su área de estudio \cite{gonzalez2003tuning}. Aunque obviamente sigue siendo muy importante el desarrollo del conocimiento específico de cada área de estudio, es un hecho que el tiempo y la atención también deben dedicarse al desarrollo de las competencias genéricas. Igualmente es importante reconocer la aplicación de dichas habilidades genéricas fuera del ámbito académico, ya que son cada vez más relevantes para la preparación de estudiantes para su futuro papel en la sociedad, en términos de empleabilidad y ciudadanía.

\subsection*{Las TIC en la educación}

En los últimos años, han sido numerosos los avances en lo que al uso de las \emph{Tecnologías de la Información y la Comunicación} (TIC) se refiere. Esto, junto con el asentamiento de internet, ha traido consigo que la sociedad se haya visto obligada a abordar cambios en su habitual modus operandi. Desde la manera en que los ciudadanos interactúan con las instituciones públicas hasta la forma en que estos se relacionan con sus amigos. Y por supuesto, también ha afectado a la educación. El campo de investigación que aborda el uso de la tecnología como parte del proceso de aprendizaje es el \emph{aprendizaje mejorado por la tecnología} (TEL, del inglés \emph{Technology Enhanced Learning}).

Según la UNESCO, es preciso que los docentes reciban los instrumentos necesarios para alcanzar los objetivos sociales y económicos que constituyen el eje de todo sistema educativo nacional. Para ello, se ha creado un conjunto de baremos internacionales que definen las competencias necesarias para impartir una enseñanza eficaz mediante el uso de las TIC: El Marco de competencias de los docentes en materia de TIC de la UNESCO~\cite{midoro2013guidelines}. Este marco tiene por objeto informar de la función de las TIC en la reforma educativa, así como ayudar a los Estados Miembros a que elaboren criterios de competencia en la materia para los docentes. Herramientas como los cursos virtuales, los wikis o los mundos virtuales son más que habituales como soporte a la docencia presencial, y en algunos casos, como ocurre con los cursos online masivos y abiertos (\emph{MOOCs}, del inglés \emph{Massive Open Online Courses}), es el único punto de encuentro entre el estudiante y el profesor.

En la mayoría de estas herramientas, la actividad de los estudiantes suele quedar registrada, es decir, accesos al sistema, envío de tareas, comentarios en foros, etc. Según \cite{Chebil:2012, Florian:2011} la recopilación de los rastros de interacción producidos por este tipo de herramientas, con un filtrado adecuado, podría ser una información muy valiosa para obtener indicadores del desempeño de los estudiantes en ciertas competencias. Las técnicas de \emph{Learning Analytics} facilitan la explotación de este tipo de información~\cite{conde2015exploring}. Cómo interactúan, cuándo lo hacen o con qué frecuencia consultan los recursos son cuestiones cuyas respuestas podrían utilizarse como indicadores de algunas competencias.

\subsection*{La educación como una ciencia de diseño}
\label{sec:dbr}

%Sin embargo, evaluar ciertas competencias genéricas es a menudo una tarea subjetiva.
Una vez que quedase recopilada la información del registro de los sistemas de aprendizaje, habría que plantear como utilizarla para obtener unas evidencias válidas para la evaluación del desempeño de los estudiantes en una o varias competencias genéricas. A menos que una competencia genérica esté directamente enlazada a una actividad específica, su evaluación requiere de la inventiva y originalidad del profesor para ser capaz, por un lado, de diseñar actividades que obliguen al estudiante a desempeñar las competencias que se quieren evaluar, y por otro lado, diseñar evaluaciones que midan el desempeño real del estudiante en dicha competencia. Comienza Diana Laurillard en su libro \emph{Teaching as a Design Science (La enseñanza como una ciencia de diseño)}~\cite{laurillard2012teaching} comparando la enseñanza con el arte. En ambas, tanto el artista como el profesor han de inspirar y entusiasmar a su audiencia. Los profesores tienen que lograr conectar con sus estudiantes y engancharlos con la temática de estudio. Pero al contrario de lo que ocurre en el arte, donde prácticamente "todo vale'', en la enseñanza hay que ceñirse a una serie de objetivos formales previamente definidos. La enseñanza no es una ciencia teórica que describe y explica algunos aspectos del mundo social y natural, sino una más cercana a otro tipo de ciencias tales como la ingenieria, la informática o la arquitectura, cuyo objetivo es hacer un mundo mejor, es decir, una \emph{ciencia del diseño}. En la ciencia del diseño se parte de la teoría, pero se construyen principios de diseño en lugar de nuevas teorías y se utilizan métodos prácticos en lugar de explicaciones. Su objetivo es ir mejorando con la práctica, basándose en principios y construyendo sobre el trabajo de otros.


\subsection*{Conclusión}

Los planes de estudio actuales demandan a los profesores la evaluación de competencias genéricas de su alumnado. Considerando la miríada de herramientas que los profesores tienen a su disposición en clase y que en estas se refleja la actividad de los estudiantes, se concluye que una posible solución sería aprovechar los indicadores obtenidos de estas herramientas como evidencias del desempeño de competencias genéricas. Para lograrlo se indican a continuación las dos principales contribuciones de esta tesis:
\begin{enumerate}
\item Un método para aplicar la ciencia del diseño en la evaluación de competencias genéricas de los estudiantes a partir de indicadores procedentes de los registros de actividades de aprendizaje
\item Herramientas informáticas que automaticen y den soporte a la aplicación del método.
\end{enumerate} 
