
\section{Objetivos y preguntas de investigación}
\label{sec:objetivos}

El principal objetivo de esta tesis es:

\bigskip
\textbf{Evaluar a los estudiantes en el desempeño de sus competencias genéricas mediante indicadores procedentes de los registros de actividades de aprendizaje}
\bigskip

Para alcanzar dicho objetivo, es necesario primero responder a diferentes preguntas de investigación. Para dar respuesta a las mismas se llevará a cabo una revisión sistemática de la literatura. Las preguntas de investigación a las que se tratará de dar respuesta son las siguientes:

\begin{itemize}
\item Q1. ¿Qué competencias se han evaluado de forma automática o asistida por ordenador a partir de la actividad de los estudiantes en los entornos virtuales?
\item Q2. ¿Qué métodos se utilizan para evaluar competencias genéricas mediante el uso de entornos virtuales?
\item Q3. ¿Qué técnicas se utilizan para evaluar competencias genéricas a partir de los registros de actividad de un entorno virtual?
\end{itemize}

Hay muchos trabajos en la literatura que abordan la evaluación de competencias genéricas de los estudiantes. De estos queremos obtener información de aquellos que buscan la automatización del proceso para facilitar la labor del docente, centrándonos sobre todo en qué competencia es la que evalúan. Es evidente que encontraremos muchos trabajos que abordan la evaluación de alguna competencia genérica con actividades manuales. Sin embargo, este tipo de trabajo sufren generalmente problemas de escalabilidad, por lo que bajo esta premisa los descartaremos en este análisis.


% ¿Qué método se utilizan para evaluar competencias? En la respuesta dejar en evidencia que los métodos basados en indicadores no abundan 

% tec. informatics automatizasas ...

% métodos, técnicas y herramientas (de mayor a menor nivel de abstracción)

% ¿Qué técnicas utilizan los métodos basados en evidencias?

% Referencia Fran

Se analizaran las técnicas empleadas para obtener las evidencias o indicadores objetivos de los entornos de aprendizaje virtual.


\bigskip
A partir de aquí el objetivo de esta investigación será proveer a los diseñadores de evaluaciones de un lenguaje para obtener de manera automatizada un conjunto de indicadores entre los que elegir para ir aplicándolos a sus procesos de evaluación segun las competencias genéricas que quieran evaluar. Esto se divide en los siguientes tres objetivos:
 
\begin{itemize}
\item O1. Escribir una relación de posibles fuentes de evidencias para el análisis de registros de actividad
\item O2. Definir un método que permita al docente obtener de manera automática un conjunto de indicadores de un entorno de aprendizaje virtual
\item O3. Definir un DSL que permita a los docentes investigar y diseñar estrategias de evaluación a partir de los registros contenidos en los entornos de aprendizaje virtual
\end{itemize}

Para comenzar a abordar estos objetivos partiremos de las evidencias utilizadas y referidas por los autores en la literatura como susceptibles de ser utilizadas en la evaluación de competencias genéricas. Estas evidencias se mapearán a un entorno de aprendizaje virtual y se implementaran para obtenerlas mediante el uso de alguna herramienta informática. A continuación, se definirá un DSL que permita al docente no únicamente obtener indicadores, sino investigar y diseñar diferentes estrategias de evaluación a partir de dichos indicadores. De esta forma el docente podrá ajustar los indicadores según el trabajo realizado por los estudiantes en el curso.

%LA EVALUACIÓN NO ES UN OBJETIVO.

%EN EL CAPÍTULO DE EVALUACIÓN HABRÁ QUE DEFINIR OBJETIVOS DE EVALUACIÓN Y MAPEARLO CON LOS OBJETIVOS ANTERIORES. tENDRÉ QUE HACER UN CUADRO PARA EL MAPEO.

\bigskip
%\textbf{O2. Evaluar los indicadores en escenarios reales de aprendizaje basados en competencias con soporte de un VLE}
\bigskip

Además, para proponer una serie de recomendaciones estos indicadores seran evaluados por docentes miembros de la comunidad UCA. Desarrollamos tres herramientas para obtener los indicadores sugeridos en esta tesis: \emph{AssessMediaWiki} para obtener indicadores procedentes de una wiki basada en MediaWiki; \emph{EvalCourse}, Un Lenguaje Específico de Dominio para obtener indicadores procedentes de los registros de la plataforma de cursos virtuales Moodle; y \emph{EvalSim}, un Lenguaje Específico de Dominio para obtener indicadores procedentes de los registros de un mundo virtual basado en OpenSim.


%-------------------

%Estrategia para la investigación realizada en los 3 objetivos.

%En el tercer objetivo: ¿Cómo evaluar DSL?

%Jordi Cabot había esceitro algo sobre evaluación de DSL
