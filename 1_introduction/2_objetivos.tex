
\section{Objetivos y preguntas de investigación}
\label{sec:objetivos}

El principal objetivo de esta tesis es demostrar que:

\bigskip
\textbf{Evaluar a los estudiantes en el desempeño de sus competencias genéricas mediante indicadores procedentes de los registros de actividades de aprendizaje}
\bigskip

Para alcanzar dicho objetivo, es necesario primero responder a diferentes preguntas de investigación. Para dar respuesta a las mismas se llevará a cabo una Revisión Sistemática de la Literatura (SLR, Systematic Literature Review). Las preguntas de investigación a las que se tratará de dar respuesta son las siguientes:

\bigskip
\textbf{Q1. ¿Cuáles son las competencias que se han evaluado de forma automática o asistida por ordenador a partir del uso de los entornos virtuales?}
\bigskip

Hay muchos trabajos en la literatura que abordan la evaluación de competencias genéricas de los estudiantes. De estos queremos obtener información de aquellos que buscan la automatización del proceso para facilitar la labor del docente, centrándonos sobre todo en qué competencia es la que evalúan. Es evidente que encontraremos muchos trabajos que abordan la evaluación de alguna competencia genérica con actividades manuales. Sin embargo, este tipo de trabajo sufren generalmente problemas de escalabilidad, por lo que bajo esta premisa los descartaremos en este análisis.

\bigskip
\textbf{Q2. ¿Qué herramientas o metodologías se utilizan para evaluar competencias mediante el uso de los entornos virtuales?}
\bigskip

Analizaremos las herramientas que se utilizan para facilitar a los docentes la evaluación de las competencias genéricas de sus alumnos.

\bigskip
\textbf{Q3. ¿Con qué técnicas se pueden obtener evidencias objetivas del desarrollo de las siguientes competencias en un entorno virtual?}
\bigskip

Se analizaran las técnicas empleadas para obtener las evidencias o indicadores objetivos de los entornos de aprendizaje virtual.

\bigskip
\textbf{Q4. ¿Son utilizados los registros de actividad de los entornos virtuales para la evaluación?}
\bigskip

Prestaremos especial atención a aquellos trabajos que hagan uso de los registros de actividad o logs de los entornos de aprendizaje virtual.

A partir de aquí el objetivo de esta investigación será proveer a los diseñadores de evaluación de un lenguaje para obtener de manera automatizada un conjunto de indicadores entre los que elegir para ir aplicándolos a sus procesos de evaluación segun las competencias genéricas que quieran evaluar. Esto se divide en los siguientes dos objetivos:
 
\bigskip
\textbf{O1. Definir un conjunto de indicadores susceptibles de ser obtenidos de los entornos de aprendiaje virtual para evaluar a los estudiantes en el desempeño de sus competencias genéricas}
\bigskip

Analizaremos las herramientas que se utilizan para facilitar a los docentes la evaluación de las competencias genéricas de sus alumnos.

\bigskip
\textbf{O2. Evaluar la idoneidad de dichos indicadores en un escenario completamente real}
\bigskip

Además, para proponer una serie de recomendaciones estos indicadores seran evaluados por docentes miembros de la comunidad UCA. Finalmente, tres herramientas fueron desarrolladas para obtener los indicadores sugeridos en esta tesis: \emph{AssessMediaWiki} para obtener indicadores procedentes de una wiki basada en MediaWiki; \emph{EvalCourse}, Un Lenguaje Específico de Dominio para obtener indicadores procedentes de los registros de la plataforma de cursos virtuales Moodle; y \emph{EvalSim}, un Lenguaje Específico de Dominio para obtener indicadores procedentes de los registros de un mundo virtual basado en OpenSim.