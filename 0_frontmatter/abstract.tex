
% Thesis Abstract -----------------------------------------------------


%\begin{abstractslong}    %uncommenting this line, gives a different abstract heading
\begin{resumen}        %this creates the heading for the abstract page
\selectlanguage{spanish}
% Pon tu resumen aquí.

%Este método surge de la oportunidad que se presenta de aprovechar la cantidad de información generada por los estudiantes en los entornos virtuales.

El desarrollo de las competencias genéricas es a día de hoy clave en la preparación de estudiantes para su futuro papel en la sociedad. Ante esta situación, los centros de enseñanza han de complementar la docencia del conocimiento específico con el entrenamiento y la evaluación del ejercicio de competencias genéricas. Esto ocurre en un contexto en el que las tecnologías de la información y la comunicación están consolidadas en todos los niveles educativos. Los docentes tienen a su disposición una miríada de entornos virtuales de aprendizaje con los que gestionar y supervisar el proceso de enseñanza de sus estudiantes.

En el análisis de la literatura se han identificado trabajos que utilizan las nuevas tecnologías como apoyo en la evaluación de competencias genéricas. Sin embargo, los métodos mayormente utilizados hasta ahora para la evaluación de dichas competencias presentan los siguientes inconvenientes: problemas de escalabilidad para realizar las evaluaciones con grupos grandes de estudiantes, notables diferencias entre las calificaciones dadas por los estudiantes en procesos de autoevaluación o evaluación entre iguales con respecto a las calificaciones que asignaría el docente, el propósito demasiado específico de algunas herramientas, el esfuerzo necesario para poner en práctica ciertas soluciones, y la aplicabilidad de unos indicadores para medir competencias fuera del contexto en el que se definieron. Por todo esto se detecta la necesidad de desarrollar un método para la evaluación asistida de competencias genéricas.

En esta tesis se propone un método de evaluación denominado \emph{design-based assessment} que propone la utilización de la información de los registros de actividad de los estudiantes en entornos virtuales de aprendizaje para diseñar indicadores con los que medir su desempeño en competencias genéricas. El método se contextualiza en la metodología de investigación \emph{design-based research}, ya que el diseño de indicadores se lleva a cabo dentro de un proceso iterativo en el que se va refinando el diseño (ciclo de contraste de hipótesis). Para poner en práctica el método se han desarrollado dos lenguajes específicos de dominio con los que facilitar el cálculo de indicadores de dos entornos virtuales de aprendizaje diferentes: sistemas de gestión de aprendizaje y mundos virtuales.

Para la evaluación del método y las herramientas se han realizado seis estudios de caso en tres entornos virtuales de aprendizaje diferentes: wikis, sistemas de gestión de aprendizaje y mundos virtuales. Además, las propuestas realizadas en esta tesis han sido evaluadas por docentes de todos los niveles educativos y pertenecientes a todas las ramas de conocimiento. Los resultados proporcionan, en primer lugar, evidencias a favor de que el método definido permite a los docentes obtener de manera automática valores para un conjunto de indicadores que el diseñador de la evaluación puede expresar a partir de los registros de actividad de los entornos virtuales de aprendizaje. En segundo lugar, también proporcionan evidencias a favor de que las herramientas desarrolladas les permitan tanto diseñar como contrastar estrategias de evaluación a partir de los registros de estos entornos.


\end{resumen}

\begin{abstracts}        %this creates the heading for the abstract page
\selectlanguage{british}
% Put your abstract or summary here.

The development of generic skills is a key issue for preparing students for their future role in society. Due to this circumstance, educational institutions need to complement the learning of specific knowledge of their subjects with training and evaluation of generic skills performance. This occurs in a context in which information and communications technology are consolidated at all educational levels. Teachers have access to a myriad of virtual learning environments that promote and evaluate the teaching of their students.

In the literature analysis, works have been identified that use new technologies to support the assessment of generic skills. However, the most commonly used methods present the following disadvantages: scalability issues for evaluations when large groups of students are involved, notable differences between the grades given by students in the process of self and peer evaluation when compared to those given by teachers, the high specificity of some tools, the great effort required to implement certain solutions and the applicability of indicators for measuring skills outside the context in which they were created. It is due to these reasons that the development of a method for an assisted evaluation of generic skills is needed.

This dissertation introduces an assessment method called \emph{design based assessment}, which proposes the use of the information registered in the virtual learning environments activity registers to design indicators that measure students’ performance in generic skills. The method is contextualized on the research methodology called  \emph{design-based research}, as the design of indicators is carried out in an iterative process in which the design is refined (hypothesis contrast cycle). Two domain-specific languages have been developed to implement this method in two different virtual learning environments: virtual courses and virtual worlds.

Six case studies were conducted for the assessment of the method and tools in three different virtual learning environments: wikis, learning management systems and virtual worlds. In order to contrast the proposals stated in this thesis, they have been assessed by teachers belonging to both all educational levels and all branches of knowledge. From the results, it can be derived that the method allows teachers to automatically obtain a set of indicators of students’ activity in virtual learning environments. Additionally, the developed tools allow them to design and contrast evaluation strategies from the registers of these environments. 


\end{abstracts}




%\end{abstractlongs}


% ---------------------------------------------------------------------- 
