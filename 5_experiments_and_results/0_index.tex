
% this file is called up by thesis.tex
% content in this file will be fed into the main document

%: ----------------------- introduction file header -----------------------
\begin{savequote}[50mm]
The logic of validation allows us to move between the two limits of dogmatism and skepticism. 
\qauthor{Paul Ricoeur}
\end{savequote}


\chapter{Evaluación}
\label{cha:Validation of the methodology}

% the code below specifies where the figures are stored
\ifpdf
    \graphicspath{{5_experiments_and_results/figures/PNG/}{5_experiments_and_results/figures/PDF/}{5_experiments_and_results/figures/}}
\else
    \graphicspath{{5_experiments_and_results/figures/EPS/}{5_experiments_and_results/figures/}}
\fi


%------------------------------------------------------------------------- 

En este capítulo se evalúa la idoneidad del método propuesto para la evaluación de competencias genéricas. Este capítulo se estructura de la siguiente manera:

\begin{itemize}
	\item En primer lugar se describen las características de los procedimientos que se han utilizado para llevar a cabo la evaluación:
		\begin{itemize}
			\item Publicaciones
			\item Cuestionarios
		\end{itemize}
	\item En segundo lugar se muestran los resultados de la aplicación en cada una de las actividades de aprendizaje para los que se ha implementado el método:
		\begin{itemize}
			\item Wikis: AMW
			\item VLE: SASQL Y EvalCourse
			\item Mundos virtuales: VWQL Y EvalSim
		\end{itemize}
	\item Finalmente se presentan las conclusiones de estas evaluaciones, tanto de los cuestionarios como de las publicaciones.
\end{itemize}

\section{Introducción}

La evaluación del método se ha realizado desde dos perspectivas. Por un lado, cada una de las implementaciones realizadas se ha tratado de mostrar tanto en congresos y revistas de relevancia en el área de las TEL. Mientras que por otro lado, se han realizado cuestionarios y entrevistas a profesionales de la enseñanza para evaluar la idoneidad de los indicadores obtenidos. A continuación se introducen ambos enfoques.

\subsection{Publicaciones}

Durante el desarrollo de este método se ha tratado de publicar en diferentes ámbitos a partir de cada una de sus implementaciones. La evaluación realizada en revistas de ímpacto y conferencias de relevancia son realizadas mediante procesos de \emph{revisión por par doble-ciego}, donde tanto los revisores como los autores son anónimos~\cite{ladron2008revision}. De esta forma se han recibido evaluaciones, opiniones y recomendaciones, que se han ido incorporando a las diferentes implementaciones y al método.

\subsection{Cuestionarios y entrevistas}

En el transcurso de esta tesis se ha presentado en varios foros el trabajo que se va desarrollando. Los profesionales de la educación que han asistido a las presentaciones han participado en cuestionarios de evaluación de la propuesta y han vertido sus opiniones y recomendaciones sobre el método en las entrevistas realizadas. Los profesionales que han participado son profesores universitarios, profesores de primaria y secundaria y personal de MediaWiki España.

\section{AMW aplicado a MediaWiki}

Integer fermentum rutrum urna at vestibulum. Vivamus ullamcorper erat in sapien dignissim pellentesque. Integer convallis fringilla dictum. In bibendum lectus eu nulla pretium volutpat. Morbi hendrerit fringilla tortor, sed gravida neque lacinia a. In risus magna, hendrerit vitae cursus ac, vehicula at eros. Aenean quis ipsum sit amet leo vestibulum cursus.

\section{EvalCourse aplicado a entornos de apredizaje virtual}

Cras placerat mattis dui quis vehicula. Nulla sit amet metus nibh, at auctor enim. Quisque congue ultricies sapien in suscipit. Fusce vitae placerat ante. Praesent aliquet urna ac elit consequat nec mattis augue faucibus. Nunc et sapien vel felis mollis sodales. Aenean molestie nulla vestibulum nisi fringilla vel euismod dolor tristique. Aenean fermentum, dolor eget tincidunt faucibus, risus lorem feugiat elit, sagittis malesuada eros ligula in odio. Pellentesque ac libero lobortis justo bibendum laoreet. Cras egestas lorem eget ligula dignissim sollicitudin. Vestibulum sit amet augue ultrices erat faucibus vestibulum. Aenean tincidunt faucibus leo, nec auctor diam bibendum a. Sed varius, mauris in pellentesque scelerisque, nisl ligula viverra erat, in eleifend tellus enim ac magna. Pellentesque quis est risus. Cras mollis feugiat auctor. Proin ac eros vitae nulla gravida varius.

\section{EvalSim aplicado a los mundos virtuales}

Morbi at augue sapien. Duis tempus quam vitae velit interdum ultricies. Vivamus laoreet lacinia elit sit amet vehicula. Ut congue diam ac magna hendrerit sed fermentum justo lacinia. Curabitur vel odio neque, quis consequat mi. Proin lobortis justo quis enim fermentum accumsan sagittis ipsum imperdiet. Proin sem felis, laoreet placerat egestas id, fringilla id mauris. Pellentesque a nisi sit amet leo consectetur gravida nec et dui. Curabitur quis hendrerit augue. Etiam sed dui nec tortor convallis fringilla. Proin tempor mattis diam nec egestas. Quisque condimentum elementum lacus ac porta. Vivamus congue, odio eu ullamcorper elementum, leo turpis tempus sem, at condimentum dolor quam eu nunc. Pellentesque eget risus ac velit aliquam sollicitudin sed et ipsum. 

\section{Evaluación}

Morbi at augue sapien. Duis tempus quam vitae velit interdum ultricies. Vivamus laoreet lacinia elit sit amet vehicula. Ut congue diam ac magna hendrerit sed fermentum justo lacinia. Curabitur vel odio neque, quis consequat mi. Proin lobortis justo quis enim fermentum accumsan sagittis ipsum imperdiet. Proin sem felis, laoreet placerat egestas id, fringilla id mauris. Pellentesque a nisi sit amet leo consectetur gravida nec et dui. Curabitur quis hendrerit augue. Etiam sed dui nec tortor convallis fringilla. Proin tempor mattis diam nec egestas. Quisque condimentum elementum lacus ac porta. Vivamus congue, odio eu ullamcorper elementum, leo turpis tempus sem, at condimentum dolor quam eu nunc. Pellentesque eget risus ac velit aliquam sollicitudin sed et ipsum. 

\section{Conclusiones}

Morbi at augue sapien. Duis tempus quam vitae velit interdum ultricies. Vivamus laoreet lacinia elit sit amet vehicula. Ut congue diam ac magna hendrerit sed fermentum justo lacinia. Curabitur vel odio neque, quis consequat mi. Proin lobortis justo quis enim fermentum accumsan sagittis ipsum imperdiet. Proin sem felis, laoreet placerat egestas id, fringilla id mauris. Pellentesque a nisi sit amet leo consectetur gravida nec et dui. Curabitur quis hendrerit augue. Etiam sed dui nec tortor convallis fringilla. Proin tempor mattis diam nec egestas. Quisque condimentum elementum lacus ac porta. Vivamus congue, odio eu ullamcorper elementum, leo turpis tempus sem, at condimentum dolor quam eu nunc. Pellentesque eget risus ac velit aliquam sollicitudin sed et ipsum. 








% ----------------------------------------------------------------------

