
% this file is called up by thesis.tex
% content in this file will be fed into the main document

%: ----------------------- introduction file header -----------------------
\begin{savequote}[50mm]
Personally, I think it does help, that it makes a beneficial difference, but the scientific literature on the subject is very messy.
\qauthor{Jeanne Petrek}
%“And upon the top of the pillars was lily work: so was the work of the pillars finished.”
%
% Bible quotes
\end{savequote}


\chapter{Estado del Arte}
\label{cha:State of the Art}

% the code below specifies where the figures are stored
\ifpdf
    \graphicspath{{2_state_of_the_art/figures/PNG/}{2_state_of_the_art/figures/PDF/}{2_state_of_the_art/figures/}}
\else
    \graphicspath{{2_state_of_the_art/figures/EPS/}{2_state_of_the_art/figures/}}
\fi


%------------------------------------------------------------------------- 

Los VLEs almacenan información de estudiantes, profesores, cursos, tareas, trabajos, etc. Estos elementos se relacionan y configuran para ofrecer al usuario una experiencia de curso virtual. Estos cursos son de gran importantacia hoy en día, siendo el soporte virtual de clases presenciales o incluso siendo el único medio donde unas clases o un curso se imparte. Las clases virtuales presentan numerosas ventajas con respecto a las clases tradicionales. Por un lado se elimina la limitación geográfica que tienen las clases tradicionales, además de que la oferta y variedad de cursos ofrecidos siempre será mayor. Para los estudiantes presenta otras ventajas como la flexibilidad de horario, permitiéndoles compatibilizar los estudios con una vida laboral sin renunciar a crecer profesionalmente y sobre todo, permitiéndoles estar en contacto permante con otros estudiantes y profesores mediante diferentes herramientas (foros, chats, ... etc.)~\cite{alAjlan:2008}.

% VENTAJAS: http://oedb.org/ilibrarian/10-advantages-to-taking-online-classes/

Pero además de todo lo anterior, un VLE almacena una gran cantidad de información que adecuadamente analizada y presentada podria ser de gran utilidad para los profesores para monitorizar el trabajo de sus estudiantes~\cite{podgorelec:2011}. Cada archivo, cada acceso o cada tarea realizada por cada estudiante queda registrada en el sistema. Por desgracia, esta información no esta siempre a disposición del profesor, y si lo está, require un filtrado para poder ser utilizada~\cite{Chebil:2012}. En \cite{fidalgo:2015}, se definen algunos indicadores para evaluar el desempeño de los estudiantes en la competencia del trabajo en equipo. Estos indicadores reflejan las interacciones de los estudiantes en el foro del VLE.

El objetivo de este capítulo es establecer la base teórica sobre la que se sustenta esta tesis doctoral. Se comenzará definiendo las preguntas de investigación, a las que se dará respuesta mediante un \emph{Estudio Sistemático de Mapeo} (\emph{SMS: Systematic Mapping Study}) aplicado a la ingenieria del software siguiendo las directrices descritas por Petersen~\cite{Petersen:2008}.

\section{Preguntas de investigación}

El objetivo principal de esta tesis doctoral es \emph{evaluar a los estudiantes en el desempeño de sus competencias genéricas mediante indicadores procedentes de los registros de actividades de aprendizaje}. Para abordar este objetivo ha de conocerse primero el estado del arte, dando respuesta para ello a diferentes preguntas de investigación. Las cuestiones habrán de dar repuesta a interrogantes tales cómo cuáles son las competencias genéricas que se han evaluado haciendo uso de la informática, asi cómo qué métodos se han utilizado  y si se están usando para este fin los registros de actividad de los entornos virtuales.

\bigskip
Por tanto, partiendo del objetivo principal, se definen las siguiente preguntas de investigación:
\begin{itemize}
\item Q1. ¿Qué competencias se han evaluado de forma automática o asistida por ordenador a partir de la actividad de los estudiantes en los entornos virtuales?
\item Q2. ¿Qué métodos se utilizan para evaluar competencias genéricas mediante el uso de entornos virtuales?
\item Q3. ¿Qué técnicas se utilizan para evaluar competencias genéricas a partir de los registros de actividad de un entorno virtual?
\end{itemize}

\section{Metodología}

\subsection{Protocolo de revisión}

La definición del protocolo de revisión requiere la realización de una serie de pasos para obtener la bibliografía de nuestro estudio. Los pasos a seguir son los siguientes:
\begin{enumerate}
\item Selección de motores de búsqueda (sección \ref{sec:MotoresBusqueda}).
\item Definición de los términos de búsqueda (sección \ref{sec:TerminosBusqueda}).
\item Determinación de los criterios de selección (sección \ref{sec:CriteriosBusqueda}).
\item Clasificación para la extracción de los datos (sección \ref{sec:EsquemaBusqueda}).
\end{enumerate}

%Comenzaremos indicando los motores de búsqueda que vamos a utilizar, qué términos de búsqueda utilizaremos en dichos motores y las herramientas de soporte a la revisión. Además se mostrarán qué criterios de inclusión de la bibliografía se siguen y el procedimiento de selección.

\subsection{Motores de búsqueda}
\label{sec:MotoresBusqueda}
Para encontrar la bibliografía, se realizarán consultas en las siguientes bibliotecas digitales: 
\begin{itemize}
\item Web of Science
\item Wiley Online Library
\item Science Direct
\item IEEE Digital Library (Xplore)
\end{itemize}

\subsection{Términos de búsqueda}
\label{sec:TerminosBusqueda}
Existen muchos términos que pueden utilizarse para referirse a la evaluación de competencias genéricas de manera automatizada o asistida. Por la naturaleza de nuestro trabajo, debemos contemplar siempre en las palabras de búsqueda los términos \emph{assessment} y \emph{generic skills} o \emph{generic competences}. Realizar la búsqueda por el término \emph{Assessment of generic skills} o \emph{assessing generic skills} nos planteaba la primera problemática, y es que el número de artículos devueltos era muy reducido. Por ejemplo, en la \emph{Wiley Online Library} la búsqueda del término exacto \emph{generic skills assessment} devolvió un único resultado. Sin embargo, debilitar la búsqueda con términos como \emph{generic competences} o \emph{generic skills} junto con la palabra \emph{assessment} daba un número de resultados muy elevado. En la misma biblioteca, buscar por los términos \emph{``generic skills`` and student and assessment} nos devolvía 609 resultados. En primera instancia se probó añadiendo términos como  \emph{E-Learning}, \emph{computer-assisted} o \emph{mobile learning}. Sin embargo, incluir términos de este tipo reducían también drásticamente el número de resultados obtenidos en la búsqueda, no llegando a obtenerse bibliografía más significativa que si no se incluyen. Por tanto, a tenor de las pruebas se decide eliminar de la búsqueda ese tipo de términos. La combinación de los términos de búsqueda empleados en la investigación, así como a los motores de búsqueda que fueron aplicados en cada una pueden comprobarse en la tabla \ref{tab:ResumenBusqueda}.

%Por otro lado, sí se incluyen acrónimos de diferentes entornos virtuales relacionados con las TEL, como son: \emph{TEL}, \emph{LMS}, \emph{ICT} (Information and Communications Technology), \emph{CBI} (Content-Based Instruction). Y tras varias pruebas, se descartan también de la búsqueda términos como `\emph{ICE} (Integrated Collaboration Environment) y \emph{CSCL} (Computer Supported Collaborative Learning), debido a que son términos que en conjunción con los términos principales de nuestra búsqueda no suelen aparecer y los resultados de estas búsquedas eran nulos. Un ejemplo de esto se refleja en una de las consultas realizadas en \emph{Scopus}, dónde los términos \emph{((``student assessment`` OR ``assessment of students``) AND (``generic skills`` OR ``generic competences``)) AND CSCL} no devolvían ningún resultado. 

\begin{table}[H]
  \begin{center}
  \begin{tabular}{| p{3cm} | p{5cm} | p{2cm} | p{3cm} |}
    \hline
    SOURCE & SEARCH TERMS & SEARCH SCOPE & PUBLICATION\\
    \hline
    \hline
    Web of Science & ((``generic competences`` OR ``generic skills``) AND assessment) & in All Fields & Journals\\
    \hline
    Wiley Online Library & ``generic competences`` AND assessment & in All Fields & Journals and Conferences\\
    \hline
    Science Direct & (``generic competences``) AND assessment) & in All Fields & Journals\\
    \hline
    IEEE Digital Library (Xplore) & ((``generic competences``) AND assessment) & in All Fields & Journals and Conferences\\
    \hline

%    Wiley Online Library & assessment AND ``generic competences`` OR ``generic skills`` AND (TEL OR ICT OR CBI) & in All Fields\\
%    World Scientific Net & ``generic competences`` OR ``generic skills`` AND assessment & Anywhere in article\\
%    Springer & (``generic skills`` OR ``generic competences``) AND  students AND (TEL OR CBI OR ICT) & All fields (Including full text)\\
%    ACM Digital Library & (assessment and ``generic skills``) and (TEL or LMS or ICT or CBI) & Any field (title, abstract, review)\\
%    ACM Digital Library & (assessment and ``generic competences``) and (TEL or LMS or ICT or CBI) & Any field (title, abstract, review)\\
%	  IEEE Digital Library (Xplore) & (((TEL or LMS or ICT or CBI) AND (``generic skills`` OR ``generic competences``)) AND assessment) & Full text and metadata\\
%    Scopus & (((TEL or LMS or ICT or CBI) AND (``generic skills`` OR ``generic competences``)) AND assessment) & All fields (Including full text)\\
    \hline
  \end{tabular}
\end{center}
\caption{Resumen de búsqueda de bibliografía}
\label{tab:ResumenBusqueda}
\end{table} 

\subsection{Criterios de selección}
\label{sec:CriteriosBusqueda}
Para determinar si un trabajo debía formar parte de nuestra selección de estudios primarios se leyó tanto el título, como el resumen y las palabras clave. En ocasiones esto no fue suficiente, siendo necesario complementar la lectura anterior con una somera la lectura del artículo completo y más detallada de la introducción y las conclusiones.
Nuestra búsqueda se centró en la localización de los trabajos que, habiendo sido obtenidos en el proceso de búsqueda anterior, vayan en línea con nuestro estudio y puedan ayudarnos a resolver las preguntas de investigación. Para ello, se realizó la proyección de los trabajos seleccionados utilizando los siguientes criterios de exclusión:
\begin{itemize}
\item Off Topic: trabajo no relacionado directamente con nuestra investigación. Son trabajos, que aún satisfaciendo los criterios de búsqueda porque de alguna forma se mencionan en el texto, su contribución no está relacionada con la temática de este estudio.
\item Unsupported Language: trabajo escrito en un lenguaje diferente al inglés o español. La mayoría de los textos son en inglés, por lo que este criterio de descarte apenas es utilizado.
\item Duplicated: trabajos cuya contribución principal está recogida en otros trabajos ya incluidos. 
\item Unread: trabajo que no ha podido ser leído. Son textos que no han sido leídos al no estar disponible para su lectura en las bibliotecas digitales a las que se tiene acceso desde la Universidad de Cádiz ni se ha podido encontrar por otros medios (petición por correo a los autores, búsqueda en otros repositorios de Internet, etc).
\end{itemize}

\subsection{Esquema para la extracción de datos}
\label{sec:EsquemaBusqueda}

Para la extracción de la información se han dividido los trabajos de acuerdo a los siguientes tres aspectos: tipo de investigación, tipo de contribución y ámbito de aplicación de la investigación. A continuación se discute esta clasificación.

\subsubsection{Tipo de investigación}
Esta clasificación hace referencia al tipo de trabajo de investigación llevado a cabo por el/los investigador/es. Existen diferentes enfoques para la clasificación de los trabajos según el tipo investigación que desarrollan. Algunos de estos sistemas de clasificación son los propuestos por Wieringa \cite{Wieringa:2005} y Hevner \cite{Hevner:2004}. Usamos el primero, ya que es el recomendado en el estudio sistemático de mapeo descrito por Petersen \cite{Petersen:2008}.
\begin{itemize}
\item Solución propuesta (\emph{proposal of solution}): se propone una solución para un problema; la solución puede ser innovadora o una extensión significativa de una técnica existente. Los posibles beneficios y la aplicabilidad de la solución se demuestran por un pequeño ejemplo o una buena línea de argumentación.
\item Validación de investigación (\emph{validation research}): las técnicas investigadas son nuevas y todavía no se han aplicado en la práctica. Estas técnicas podrían ser por ejemplo los experimentos, es decir, el trabajo realizado en un laboratorio.
\item Evaluación de la Investigación (\emph{evaluation research}): las técnicas se aplican en la práctica y se lleva a cabo una evaluación de la técnica. Se muestra cómo se implementa la técnica en la práctica (implementación de la solución) y cuáles son las consecuencias de la aplicación en términos de ventajas y desventajas (evaluación de implementación).
\item Artículos de Experiencia (\emph{experience papers}): trabajos que explican qué y cómo algo se ha llevado a cabo en la práctica. Basado en la experiencia personal del autor.
\item Artículos de opinión (\emph{opinion papers}): estos trabajos expresan la opinión personal de alguien acerca de la bondad o viabilidad de una determinada técnica, o cómo se deben realizar las cosas. No se basan en metodologías de trabajo y de investigación relacionadas.
\item Trabajos filosóficos (\emph{philosophical papers}): estos trabajos esbozan una nueva forma de ver las cosas existentes, estructurando el campo en forma de una taxonomía o un marco conceptual.
\end{itemize}

\subsubsection{Tipo de contribución}
En este apartado se clasifican los trabajos según el tipo de contribución que realizan estos al ámbito en el que se desarrollan. Una vez realizado el estudio sistemático de la literatura y habiendo seleccionado los artículos, se realiza una clasificación en base a la aportación de éstos. El uso de algunos términos puede ser confuso, debido a la interpretación que hace el autor del mismo. Algunos de estos términos son framework, modelo, estrategia, proceso, procedimiento, método o metodología. Nuestra clasificación es la siguiente:
\begin{itemize}
\item Modelo (\emph{model}): es una representación de procesos, modelos o sistemas pertenecientes a un supra-sistema, cuyo fin es el análisis de interacción de ellos para mantener una relación flexible que les permita cumplir su función particular y cumplir la función de dicho supra-sistema.
\item Proceso (\emph{process}): contempla aquellos trabajos cuya contribución sea descrita por los autores como una serie de pasos.
\item Herramienta (\emph{tool}): se utiliza para los artículos que presentan un software independiente o una extensión de algún otro programa.
\item Framework (\emph{framework}): aquí se consideran aquellos trabajos que contribuyen con una combinación de los elementos anteriores (es decir, con un modelo, un proceso y una herramienta).
\item Técnica (\emph{technique}): un procedimiento utilizado para llevar a cabo una actividad o tarea específica. Podría venir acompañado de una herramienta de apoyo.
\end{itemize}

\subsubsection{Ámbito de aplicación de la investigación}
Además de los clasificaciones anteriores, es necesario recoger más información acerca los conceptos que representan la contribución de la investigación. Para ello se recoge información sobre el ámbito de la evaluación de competencias sobre el que se aplica cada contribución. Una vez recogida esta información, se agrupan según sus similitudes, quedando finalmente la siguiente clasificación:
\begin{itemize}
\item Resultados de aprendizaje del curso y rúbricas (\emph{CLO and rubrics}): los resultados de aprendizaje del curso se evalúan mediante rúbricas o plantillas de evaluación que miden el rendimiento de los alumnos. Esto proporciona al docente un indicador de sus logros de aprendizaje de cada alumno. Las rúbricas pueden estar o no en soporte informático, pero generalmente no aprovechan la tecnología para automatizar tareas.
\item Evaluación entre iguales y autoevaluación (\emph{peer and self eAssessment}): uno de los problemas con los que se encuentran los profesores es la escalabilidad de la tarea de evaluación de competencias cuando el grupo de alumnos es grande. Hay un gran conjunto de trabajos, que aunque se apoyen en la tecnología para realizar alguna actividad, tienen el problema de que la evaluación ha de ser manual. En estos caso, mediante la autoevaluación o evaluación entre iguales los estudiantes se evalúan. De esta manera no sólo descargan de trabajo al profesor haciendo esta evaluación, sino que además se fomenta la capacidad crítica y de análisis del alumno.
\item Aprendizaje basado en juegos (\emph{GBL}): el aprendizaje basado en juegos se sirve de juegos que están diseñados expresamente para enseñar al usuario acerca de ciertos temas, ampliar conceptos o reforzar el desarrollo o aprendizaje de una habilidad mientras juegan. En ellos los alumnos tienen que completar diferentes pruebas o fases obteniendo puntos en cada una de ellas. Por cada prueba o fase superada, el jugador, o alumno en este caso, obtendrá una serie de puntos. Se podrá decir que un alumno ha alcanzado el nivel de madurez necesario en una competencia si alcanza una predefinida puntuación.
\item E-Evaluación y revisiones (\emph{eAssessment and reviews}): trabajos en los que se obtienen indicadores del desempeño de estudiantes en una o varias competencias de manera automática mediante el uso de algún software. Además se muestran otros trabajos sobre la situación actual en la evaluación de competencias genéricas, su importancia actual y sobre un conjunto de técnicas, metodologías o herramientas que se han desarrollado y utilizado.
\end{itemize}

\subsection{Visualización y análisis de los datos}
Tras obtener los estudios primarios, hay una etapa de análisis, donde se resumen los datos extraídos para así responder a las preguntas de investigación planteadas. El análisis de los resultados se centra en el estudio de las publicaciones para cada categoría y por lo tanto, la determinación del grado de cobertura de cada categoría. Esta información generalmente se resume en tablas y/o gráficos. Otro método utilizado en nuestro estudio es la combinación de diferentes categorías (por ejemplo, el ámbito de investigación contra el tipo contribución) y mostrarlos en un mapa sistemático en la forma de un gráfico de burbujas.
En el siguiente capítulo se mostrarán los resultados obtenidos.

\section{Resultados}

\section{Respuestas}

%\chapter{Trabajo en curso y futuro}

\section{Conclusiones}







% ----------------------------------------------------------------------

